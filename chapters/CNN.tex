\textit{Convolutional Neural Network} (CNN) merupakan salah satu jenis neural network yang sering digunakan pada data gambar. CNN dapat digunakan untuk merekognisi suatu objek pada gambar.
Secara umum CNN hampir sama dengan neural network pada umumnya. CNN terdiri dari neuron yang memiliki \textit{weight}, bias serta \textit{activation function}.

\section{Apa itu \textit{Convolutional Neural Network} (CNN)?}
Pada tutorial ini, kita akan mempelajari dasar-dasar CNN. Namun sebelumnya kita akan membahas pertanyaan-pertanyaan dibawah terlebih dahulu.
\begin{enumerate}
	\item Bagaimana cara otak kita bekerja?
	\item Bagaimana cara kerja CNN?
	\item Bagaimana CNN dapat memindai gambar?
	\item Bagaimana \textit{Neural Network} membaca ekspresi wajah?
	\item Apa saja langkah-langkah dalam proses CNN?
\end{enumerate}

\textbf{Pertanyaan pertama: Bagaimana cara kerja otak manusia?}

Lebih tepatnya, bagaimana cara kita mengenali benda-benda dan orang-orang di sekitar kita baik secara langsung ataupun dari sebuah gambar? Memahami hal ini merupakan bagian besar dalam memahami CNN. Singkatnya, otak kita bergantung pada pendeteksian fitur/ciri dan otak akan mengkategorikan objek yang kita lihat.

Sebagai contoh, Anda mungkin telah melalui ratusan situasi dan kondisi dalam hidup Anda. Dimana Anda melihat sesuatu secara instan, berhasil menjadi sesuatu, dan kemudian setelah melihatnya lebih teliti, Anda menyadari bahwa itu sebenarnya adalah sesuatu yang sangat berbeda.

Apa yang terjadi di sana adalah bahwa otak Anda mendeteksi objek untuk pertama kalinya, tetapi karena tampilan itu singkat, otak Anda tidak dapat memproses cukup banyak fitur objek sehingga dapat dikategorikan dengan benar.
%tobecontinue\m/