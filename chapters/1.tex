\section{Pendahuluan}
\subsection{Sejarah Deep Learning}
Deep Learning mulai diperkenalkan di tahun 2006. Pada tahun 2006, Geoffrey Hinton memperkenalkan salah satu varian jaringan saraf tiruan yang disebut deep belief nets, ide untuk men-train model jaringan saraf tiruan ini adalah dengan men-train dua layer kemudian tambahkan satu layer diatasnya, kemudian train hanya layer teratas dan begitu seterusnya. Dengan strategi ini dapat men-train model jaringan saraf tiruan dengan layer lebih banyak dari model-model sebelumnya. 

Setelah istilah deep learning populer, deep learning belum menjadi daya tarik yang besar bagi para peneliti karena jaringan saraf tiruan dengan banyak layer memiliki kompleksitas algoritma yang besar, sehingga membutuhkan komputer dengan spesifikasi tinggi, dan tidak efisien secara komputasi saat itu. Hingga pada tahun 2009 penggunaan GPU untuk deep learning diperkenalkan melalui paper yang berjudul Large-scale Deep Unsupervised Learning using Graphics Processors. Dengan menggunakan GPU jaringan saraf tiruan dapat berjalan lebih cepat dibanding dengan menggunakan CPU. Dengan tersedianya hardware yang memadai perkembangan deep learning mulai pesat, dan menghasilkan produk-produk yang dapat kita nikmati saat ini seperti pengenal wajah, self-driving car, pengenal suara, dan lain lain.

\subsection{Apa itu Deep Learning}
Deep Learning (Pembelajaran Dalam) atau sering dikenal dengan istilah Pembelajaran Struktural Mendalam (Deep Structured Learning) atau Pembelajaran Hierarki (Hierarchical learning) adalah salah satu cabang dari ilmu pembelajaran mesin (Machine Learning) yang terdiri algoritma pemodelan abstraksi tingkat tinggi pada data menggunakan sekumpulan fungsi transformasi non-linear yang ditata berlapis-lapis dan mendalam.

Deep Learning adalah salah satu jenis algoritma jaringan saraf tiruan yang menggunakan metadata sebagai input dan mengolahnya menggunakan sejumlah lapisan tersembunyi (hidden layer) transformasi non linier dari data masukan untuk menghitung nilai output. Algortima pada Deep Learning memiliki fitur yang unik yaitu sebuah fitur yang mampu mengekstraksi secara otomatis. Hal ini berarti algoritma yang dimilikinya secara otomatis dapat menangkap fitur yang relevan sebagai keperluan dalam pemecahan suatu masalah.
