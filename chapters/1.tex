\section{Pendahuluan}
\subsection{Apa itu Deep Learning}
Deep Learning (Pembelajaran Dalam) atau sering dikenal dengan istilah Pembelajaran Struktural Mendalam (Deep Structured Learning) atau Pembelajaran Hierarki (Hierarchical learning) adalah salah satu cabang dari ilmu pembelajaran mesin (Machine Learning) yang terdiri algoritma pemodelan abstraksi tingkat tinggi pada data menggunakan sekumpulan fungsi transformasi non-linear yang ditata berlapis-lapis dan mendalam. \par

Deep Learning adalah salah satu jenis algoritma jaringan saraf tiruan yang menggunakan metadata sebagai input dan mengolahnya menggunakan sejumlah lapisan tersembunyi (hidden layer) transformasi non linier dari data masukan untuk menghitung nilai output. Algortima pada Deep Learning memiliki fitur yang unik yaitu sebuah fitur yang mampu mengekstraksi secara otomatis. Hal ini berarti algoritma yang dimilikinya secara otomatis dapat menangkap fitur yang relevan sebagai keperluan dalam pemecahan suatu masalah. \par


